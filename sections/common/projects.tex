\skills{\href{https://github.com/Cinnamon/kotaemon}{Kotaemon} (\textmd{14.7k+ \scriptsize \faStar}) - An open-source tool for
local RAG application. \hfill Jan 2024 --- May 2024}\\
\subtext{
    Open-source project, Co-creator
    \hfill
    \href{https://cinnamon.github.io/kotaemon/}{\faFileLines}
    \href{https://github.com/Cinnamon/kotaemon}{\faGithub}
}
\begin{zitemize}
    \item A local RAG-based tool for chatting with your documents. Built with both end users and developers in mind.
    \item For end users: A local Question Answering UI for RAG-based QA.
    \item For developers: A framework for building your own RAG-based QA pipeline.
\end{zitemize}

\skills{A local application for chatting PDF documents \hfill Jan 2024 --- Feb 2024}\\
\subtext{
    Personal project
    \hfill
    \href{https://lone17.github.io/docqa/}{\faFileLines}
    \href{https://github.com/lone17/docqa}{\faGithub}
}
\begin{zitemize}
    \item A Retrieval Augmented Generation (RAG) application for question answering on PDF
    documents.
    \item Complete pipeline from PDF parsing to indexing, retrieval and generation.
    \item Instead of continue working on this, I moved to build \href{https://github.com/Cinnamon/kotaemon}{kotaemon}.
\end{zitemize}


\skills{Data Utility Improvement Experiment for DECAF \hfill Oct 2022 --- Nov 2022}\\
\subtext{
    Personal research
    \hfill
    \href{https://github.com/lone17/DECAF/blob/main/main.pdf}{\faFileLines}
    \href{https://github.com/lone17/DECAF}{\faGithub}
}
\begin{zitemize}
    \item A personal research on Causal Inference, Algorithmic Fairness and specifically the paper \href{https://arxiv.org/abs/2110.12884}{DECAF: Generating Fair Synthetic Data Using Causally-Aware Generative networks}.
    \item Conducted experiments on improving data utility of the DECAF method using alternating graph during synthesis while still achieving similar level of  fairness.
    \item Gave discussion and suggestions on the choice of data utility metrics.
\end{zitemize}

\skills{Channel-invariant Deformable Convolution \hfill Feb 2020 - May 2020}\\
\subtext{
    A part of my Undergrade Thesis
    \hfill
    \href{https://github.com/lone17/deform-conv/blob/master/deform_conv/layers.py}{\faGithub}
}
\begin{zitemize}
    \item A modified version of Deformable Convolution where the convolution offsets
    stay the same for all channels.
    \item Sped up the Deformable Convolution operation by an order of magnitude while still achieving similar performance.
\end{zitemize}

\skills{Gender/Accent Classification for Vietnamese short voice recordings \hfill Aug 2018 --- Sep 2018}\\
\subtext{
    \href{https://challenge.zalo.ai}{Zalo AI Challenge 2018 \faUpRightFromSquare}
    \hfill
    \href{https://github.com/lone17/speech-processing}{\faGithub}
}
\begin{zitemize}
    \item Problem description: Classify the speaker's voice in a recording (typically
    under 3 seconds) by gender (male/female) and regional accent (northern/central/southern).
    \item \textbf{4th place} on the Private Leaderboard, achieved \textbf{79.208\%
        accuracy} within 10 days as an individual participant.
    \item About the competition: Zalo AI Challenge is an annual Kaggle-like competition hosted
    by Zalo - one of the biggest tech companies in Vietnam. In 2018, the competition attracted
    over 700 teams competed in 3 challenges.
\end{zitemize}

% \skills{Electric Meter OCR \hfill Oct 2019 --- Nov 2019}\\
% \subtext{
%     University Coursework Project
%     \hfill
%     \href{https://github.com/lone17/electric-meter-ocr/blob/master/Report.pdf}{\faFileLines}
%     \href{https://github.com/lone17/electric-meter-ocr}{\faGithub}
% }
% \begin{zitemize}
%     \item Extract the value on the dial from images of electric meters. The solution is meant to be used in embedded hardware.
%     \item Achieved \textbf{0.08 on edit distance} with total code size
%     \textbf{under 10MB} and processing time \textbf{under 0.3s/image}.
% \end{zitemize}



